\documentclass{article}

\usepackage{geometry}
\usepackage{mathtools}
\usepackage{graphicx}
\usepackage[hidelinks]{hyperref}
\usepackage{cleveref}
\usepackage{tikz}
\usepackage{pgfplots}

\pgfplotsset{
  compat=1.13,
  every axis/.style={
    legend cell align=left
  }
}

\newcommand{\mail}[1]{
  \href{mailto:#1}{#1}
}

\geometry{
  top=2cm,
  bottom=2cm,
  right=3cm,
  left=3cm
}

\begin{document}

\begin{center}
  \textbf{
    \LARGE Deep Learning with Stacked AEs \& RBMs \\
    \vspace{.5ex}
    \large DD2437 - Artificial Neural Networks \& Deep Architectures - Lab 4\\
    \vspace{1ex}
    \large
    \begin{tabular}{ccc}
      Niels Agerskov & Lukas Bjarre & Gabriel Carrizo \\
      \mail{agerskov@kth.se} & \mail{lbjarre@kth.se} & \mail{gabcar@kth.se}
    \end{tabular}
  } \\
  \vspace{.5ex}
  \rule{\textwidth}{0.4pt}
\end{center}

This lab will examine two different artificial neural network structures,
Auto Encodes (AE) and Restricted Boltzmann Machines (RBM).
Their effectiveness in a learning task 
and the effect of the layer depth of the models
will be tested and evaluated.

\section{Feature learning}
In this first task shallow versions of both models are trained
as benchmarks for the later deeper versions.
The dataset used is a subset of the MNIST dataset
containing $28 \times 28$ images of handwritten digits from 0 to 9
together with correct labels of the written digit.
All the pixel values has for simplicity's sake been converted to binary values
via simple thresholding.

\begin{figure}[!ht]
  \centering
  \begin{tikzpicture}
    \begin{axis}[
      width=0.8\textwidth,
      height=200pt,
      title=\textbf{RBM training results},
      xlabel={Epoch},
      ylabel={Validation error},
      every axis plot/.style={
        no marks,
        line width=0.8pt
      }
    ]
      \foreach \color/\hiddenunits in {10/50, 40/75, 70/100, 100/150} { 
        \edef\temp{
          \noexpand\addplot +[
            color=red!\color!blue
          ]
          table [
            x index=0,
            y index=2,
            col sep=comma
          ]
          {../data/rbm_hidden_20e_\noexpand\hiddenunits.csv};
          \noexpand\addlegendentry{$n_{\text{hidden}} = \hiddenunits$}
        }
        \temp
      }
    \end{axis}
  \end{tikzpicture}
\end{figure}

\end{document}

